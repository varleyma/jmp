%%%%%%%%%%%%%%%%%%%%%%%%%%%%%%%%%%%%%%%%%%%%%%%%%%%%%%%%%%%%%%%%%%%%%%%%
%%%%%%%%%%%%%%%%%%%%%% Simple LaTeX CV Template %%%%%%%%%%%%%%%%%%%%%%%%
%%%%%%%%%%%%%%%%%%%%%%%%%%%%%%%%%%%%%%%%%%%%%%%%%%%%%%%%%%%%%%%%%%%%%%%%

%%%%%%%%%%%%%%%%%%%%%%%%%%%%%%%%%%%%%%%%%%%%%%%%%%%%%%%%%%%%%%%%%%%%%%%%
%% NOTE: If you find that it says                                     %%
%%                                                                    %%
%%                           1 of ??                                  %%
%%                                                                    %%
%% at the bottom of your first page, this means that the AUX file     %%
%% was not available when you ran LaTeX on this source. Simply RERUN  %%
%% LaTeX to get the ``??'' replaced with the number of the last page  %%
%% of the document. The AUX file will be generated on the first run   %%
%% of LaTeX and used on the second run to fill in all of the          %%
%% references.                                                        %%
%%%%%%%%%%%%%%%%%%%%%%%%%%%%%%%%%%%%%%%%%%%%%%%%%%%%%%%%%%%%%%%%%%%%%%%%

%%%%%%%%%%%%%%%%%%%%%%%%%%%% Document Setup %%%%%%%%%%%%%%%%%%%%%%%%%%%%

% Don't like 10pt? Try 11pt or 12pt
\documentclass[10pt]{article}

% This is a helpful package that puts math inside length specifications
\usepackage{calc}
%\usepackage[none]{hyphenat}


% Simpler bibsection for CV sections
% (thanks to natbib for inspiration)
\makeatletter
\newlength{\bibhang}
\setlength{\bibhang}{1em}
\newlength{\bibsep}
 {\@listi \global\bibsep\itemsep \global\advance\bibsep by\parsep}
\newenvironment{bibsection}%
        {\vspace{-\baselineskip}\begin{list}{}{%
       \setlength{\leftmargin}{\bibhang}%
       \setlength{\itemindent}{-\leftmargin}%
       \setlength{\itemsep}{\bibsep}%
       \setlength{\parsep}{\z@}%
        \setlength{\partopsep}{0pt}%
        \setlength{\topsep}{0pt}}}
        {\end{list}\vspace{-.6\baselineskip}}
\makeatother

% Layout: Puts the section titles on left side of page
\reversemarginpar

%
%         PAPER SIZE, PAGE NUMBER, AND DOCUMENT LAYOUT NOTES:
%
% The next \usepackage line changes the layout for CV style section
% headings as marginal notes. It also sets up the paper size as either
% letter or A4. By default, letter was used. If A4 paper is desired,
% comment out the letterpaper lines and uncomment the a4paper lines.
%
% As you can see, the margin widths and section title widths can be
% easily adjusted.
%
% ALSO: Notice that the includefoot option can be commented OUT in order
% to put the PAGE NUMBER *IN* the bottom margin. This will make the
% effective text area larger.
%
% IF YOU WISH TO REMOVE THE ``of LASTPAGE'' next to each page number,
% see the note about the +LP and -LP lines below. Comment out the +LP
% and uncomment the -LP.
%
% IF YOU WISH TO REMOVE PAGE NUMBERS, be sure that the includefoot line
% is uncommented and ALSO uncomment the \pagestyle{empty} a few lines
% below.
%

%% Use these lines for letter-sized paper
\usepackage[paper=letterpaper,
            %includefoot, % Uncomment to put page number above margin
            marginparwidth=1.2in,     % Length of section titles
            marginparsep=.05in,       % Space between titles and text
            margin=1in,               % 1 inch margins
            includemp]{geometry}

%% Use these lines for A4-sized paper
%\usepackage[paper=a4paper,
%            %includefoot, % Uncomment to put page number above margin
%            marginparwidth=30.5mm,    % Length of section titles
%            marginparsep=1.5mm,       % Space between titles and text
%            margin=25mm,              % 25mm margins
%            includemp]{geometry}

%% More layout: Get rid of indenting throughout entire document
\setlength{\parindent}{0in}

%% This gives us fun enumeration environments. compactitem will be nice.
\usepackage{paralist}

%% Reference the last page in the page number
%
% NOTE: comment the +LP line and uncomment the -LP line to have page
%       numbers without the ``of ##'' last page reference)
%
% NOTE: uncomment the \pagestyle{empty} line to get rid of all page
%       numbers (make sure includefoot is commented out above)
%
\usepackage{fancyhdr,lastpage}
\pagestyle{fancy}
%\pagestyle{empty}      % Uncomment this to get rid of page numbers
\fancyhf{}\renewcommand{\headrulewidth}{0pt}
\fancyfootoffset{\marginparsep+\marginparwidth}
\newlength{\footpageshift}
\setlength{\footpageshift}
          {0.5\textwidth+0.5\marginparsep+0.5\marginparwidth-2in}
\lfoot{\hspace{\footpageshift}%
       \parbox{4in}{\, \hfill %
                    \arabic{page} of \protect\pageref*{LastPage} % +LP
%                    \arabic{page}                               % -LP
                    \hfill \,}}

% Finally, give us PDF bookmarks
\usepackage{color,hyperref}
\definecolor{darkblue}{rgb}{0.0,0.0,0.3}
\hypersetup{colorlinks,breaklinks,
            linkcolor=darkblue,urlcolor=darkblue,
            anchorcolor=darkblue,citecolor=darkblue}

%%%%%%%%%%%%%%%%%%%%%%%% End Document Setup %%%%%%%%%%%%%%%%%%%%%%%%%%%%


%%%%%%%%%%%%%%%%%%%%%%%%%%% Helper Commands %%%%%%%%%%%%%%%%%%%%%%%%%%%%

% The title (name) with a horizontal rule under it
%
% Usage: \makeheading{name}
%
% Place at top of document. It should be the first thing.
\newcommand{\makeheading}[1]%
        {\hspace*{-\marginparsep minus \marginparwidth}%
         \begin{minipage}[t]{\textwidth+\marginparwidth+\marginparsep}%
                {\large \bfseries #1}\\[-0.15\baselineskip]%
                 \rule{\columnwidth}{1pt}%
         \end{minipage}}

% The section headings
%
% Usage: \section{section name}
%
% Follow this section IMMEDIATELY with the first line of the section
% text. Do not put whitespace in between. That is, do this:
%
%       \section{My Information}
%       Here is my information.
%
% and NOT this:
%
%       \section{My Information}
%
%       Here is my information.
%
% Otherwise the top of the section header will not line up with the top
% of the section. Of course, using a single comment character (%) on
% empty lines allows for the function of the first example with the
% readability of the second example.
\renewcommand{\section}[2]%
        {\pagebreak[2]\vspace{1.3\baselineskip}%
         \phantomsection\addcontentsline{toc}{section}{#1}%
         \hspace{0in}%
         \marginpar{
         \raggedright \scshape #1}#2}
         
         
         \newcommand{\astfootnote}[1]{
\let\oldthefootnote=\thefootnote
\setcounter{footnote}{0}
\renewcommand{\thefootnote}{\fnsymbol{footnote}}
\footnote{#1}
\let\thefootnote=\oldthefootnote
}

% An itemize-style list with lots of space between items
\newenvironment{outerlist}[1][\enskip\textbullet]%
        {\begin{itemize}[#1]}{\end{itemize}%
         \vspace{-.6\baselineskip}}

% An environment IDENTICAL to outerlist that has better pre-list spacing
% when used as the first thing in a \section
\newenvironment{lonelist}[1][\enskip\textbullet]%
        {\vspace{-\baselineskip}\begin{list}{#1}{%
        \setlength{\partopsep}{0pt}%
        \setlength{\topsep}{0pt}}}
        {\end{list}\vspace{-.6\baselineskip}}

% An itemize-style list with little space between items
\newenvironment{innerlist}[1][\enskip\textbullet]%
        {\begin{compactitem}[#1]}{\end{compactitem}}

% An environment IDENTICAL to innerlist that has better pre-list spacing
% when used as the first thing in a \section
\newenvironment{loneinnerlist}[1][\enskip\textbullet]%
        {\vspace{-\baselineskip}\begin{compactitem}[#1]}
        {\end{compactitem}\vspace{-.6\baselineskip}}

% To add some paragraph space between lines.
% This also tells LaTeX to preferably break a page on one of these gaps
% if there is a needed pagebreak nearby.
\newcommand{\blankline}{\quad\pagebreak[2]}

% Uses hyperref to link DOI
\newcommand\doilink[1]{\href{http://dx.doi.org/#1}{#1}}
\newcommand\doi[1]{doi:\doilink{#1}}

% Define \cvdoublecolumn, which sets its arguments in two columns without any labels

\newcommand{\cvdoublecolumn}[2]{%
  \cvitem[0.75em]{}{%
    \begin{minipage}[t]{\listdoubleitemcolumnwidth}#1\end{minipage}%
    \hfill%
    \begin{minipage}[t]{\listdoubleitemcolumnwidth}#2\end{minipage}%
    }%
}

% usage: \cvreference{name}{address line 1}{address line 2}{address line 3}{address line 4}{e-mail address}{phone number}
% Everything but the name is optional
% If \addresssymbol, \emailsymbol or \phonesymbol are specified, they will be used.
% (Per default, \addresssymbol isn't specified, the other two are specified.)
% If you don't like the symbols, remove them from the following code, including the tilde ~ (space).

\newcommand{\cvreference}[7]{%
    \textbf{#1}\newline% Name
    \ifthenelse{\equal{#2}{}}{}{\addresssymbol~#2\newline}%
    \ifthenelse{\equal{#3}{}}{}{#3\newline}%
    \ifthenelse{\equal{#4}{}}{}{#4\newline}%
    \ifthenelse{\equal{#5}{}}{}{#5\newline}%
    \ifthenelse{\equal{#6}{}}{}{\emailsymbol~\texttt{#6}\newline}%
    \ifthenelse{\equal{#7}{}}{}{\phonesymbol~#7}}


%%%%%%%%%%%%%%%%%%%%%%%% End Helper Commands %%%%%%%%%%%%%%%%%%%%%%%%%%%

%%%%%%%%%%%%%%%%%%%%%%%%% Begin CV Document %%%%%%%%%%%%%%%%%%%%%%%%%%%%

\begin{document}
\makeheading{Michael Varley}





\section{Contact Information}
%
% NOTE: Mind where the & separators and \\ breaks are in the following
%       table.
%
% ALSO: \rcollength is the width of the right column of the table
%       (adjust it to your liking; default is 1.85in).
%
\newlength{\rcollength}\setlength{\rcollength}{2.5in}%
%
%\begin{tabular}[t]{@{}p{\textwidth-\rcollength}p{\rcollength}}
550 S Limestone  \hfill \href{mailto:michael.varley@uky.edu}{michael.varley@uky.edu}  \\
Lexington, KY 40506 \hfill \href{https://www.michaelvarley.com}{www.michaelvarley.com}
 %\multicolumn{1}{r}{ \href{https://www.chicagobooth.edu/faculty/directory/y/constantine-yannelis}{https://www.chicagobooth.edu/}\hspace{.6cm}\\
%5807 S Woodlawn Ave     &\\
%Chicago, IL, 60637  & \\
%\end{tabular}

%\section{Citizenship}
%
%USA

\section{Academic Appointments}
%
\href{https://gatton.uky.edu/}%
{{\textbf{University of Kentucky \\
	Gatton College of Business and Economics}}} \\
%Chicago, IL, USA\\
Assistant Professor of Finance
             \hfill 2024-Present \\

\section{Education}
%
\href{https://www.uchicago.edu/}%
{{\textbf{University of Chicago}}} \\
%Chicago, IL, USA\\
Ph.D.,
        \href{https://financialeconomics.uchicago.edu/}
             {Joint Program in Financial Economics} 
             \hfill 2018-2024 \\
M.A.,
        \href{https://economics.uchicago.edu/}
             {Economics} 
             \hfill 2016-2017 \\ 

             
\href{https://www.ufl.edu/}{{\textbf{University of Florida}}} \\
%Gainesville, FL, USA
B.A.,
        \href{https://www.ufl.edu/}
             {Economics \& Mathematics} 
             \hfill 2010-2014 \\

\section{Research Interests} Real Estate, Banking, Corporate Finance, Household Finance \\

\section{Working Papers}  \textbf{Contractual Lock-In: Mortgage Prepayment Penalties and Mobility}  \\

\textbf{Abstract:} Do prepayment penalties lock-in borrowers, reducing mobility? Using monthly mortgage performance and borrower panel data, I study mobility before and after the expiration date of prepayment penalties. I find that borrowers increase mobility once the penalty expires: a penalty expiration leads to a 38\% higher moving rate post-expiration relative to baseline moving rates in my sample. These expiration induced moves are to places with higher economic opportunities: the effect is just as strong for small and long distance moves and the moves are disproportionately to zipcodes with high levels of average income and high levels of upward income mobility. I then study what can explain these effects on mobility. I find that housing equity at the time the penalty expires is an important input into the mobility response of borrowers: very low and underwater LTV borrowers responses are muted, while most of the response comes from high, but $<100$ LTV borrowers. I interpret this finding to be consistent with credit constraints stemming from the housing market being an important financial friction to household mobility. These results imply that mortgage contract features can interact with credit market imperfections that result in large frictions to moves that would otherwise improve economic circumstances, even when borrowers are above water on their mortgage. \\

\textbf{The Unintended Consequences of Deregulation: \\
Evidence from the Savings and Loan Industry} \\

\textbf{Abstract:} Regulating the balance sheets of financial intermediaries can lead to misallocation of capital that in the long run leads to suboptimal credit provision. However, large scale changes to regulatory frameworks can lead to massive disruptions in credit provision in the short run that can unravel an industry that deregulation was meant to help. To see how disruptive these short-run forces can be, I exploit the differential exposure the 1970s U.S. savings and loan (S\&L) lenders had to the partial repeal of the government-imposed interest rate ceilings on their deposit accounts and find that even small increases in lender's cost of funding leads to relatively large drops in mortgage origination--a 2SLS estimate gives a policy induced increase in deposit rates of 1 percentage point (14\% increase from pre-period baseline) leads to a 40-50 percent drop in mortgage lending, or about a 3 percent drop in mortgage lending for each 1 percent increase in deposit rates. To help interpret the results, I build a simple banking model similar to Tobin (1982) and Kashyap and Stein (1995) to see how deregulation can lead to large quantity responses in a lender's optimization problem, especially in an environment where the marginal cost of funding is increasing dramatically and the marginal revenue of loans adjusts relatively slowly. Despite a simple model, it provides a rich set of predictions that are borne out in the data. My research suggests that the timing of a policy can be an important determinant to its effectiveness.

\section{Works in Progress}  \textbf{Reclassification Risk in Mortgage Markets? \\
Evidence from Prepayment Penalty Bans} \\

\textbf{Abstract:} I exploit variation induced by state bans of prepayment penalties on mortgages of a certain loan balance to study the extent mortgage markets are exposed to reclassification risk. Using difference-in differences and difference-in-discontinuities designs with administrative loan servicing data, I find that these bans were effective: the fraction of loans with a prepayment penalty decreases by around 70 to 80\%. I also find, however, that interest rates and credit provision are largely unaffected by the ban: effects are statistically indistinguishable from zero and I can reject small levels of passthrough. Importantly, I also find no evidence of heterogeneous treatment across borrower risk profile: low credit score and high leverage individuals are most exposed to treatment, yet experience similar levels of (small) passthrough. I provide suggestive evidence that the lack of salience of these penalties may have inhibited them from being effective tools to insure reclassification risk. \\

\section{Awards and Grants}  Stigler Center PhD Dissertation Award and Bradley Fellow \hfill 2023-2024 \\
Rosen Memorial PhD Fellowship \hfill 2023-2024 \\
John and Serena Liew Fama-Miller Research Grant \hfill 2022-2024 \\
Financial Economics PhD Program Fellowship \hfill 2019-2023 \\
Kenneth C. Griffin Department of Economics PhD Fellowship \hfill 2018-2019 \\
Neubauer PhD Fellowship \hfill 2018-2019 \\

\section{Teaching Experience} PhD Corporate Finance II (TA for Amir Sufi) \hfill 2021-2023 \\
MBA Financial Markets and Institutions (TA for Doug Diamond)  \hfill 2021-2022 \\

\section{Conferences and Seminars} \textbf{2024}: University of Chicago Stigler Lunch, University of Kentucky (Gatton), Georgia Institute of Technology (Scheller), Florida State University (College of Business), Pennsylvania State University (Smeal), Indiana University (Kelley) and Chicago Fed Conference On Housing Affordability, Office Real Estate, And Remote Work (Student Session). \\

\textbf{2023}: Amir Sufi's Corporate Finance Reading Group, Chicago Booth Finance Brownbag, University of Chicago (Booth Faculty Seminar) \\

\textbf{2022} Amir Sufi's Corporate Finance Reading Group, Chicago Booth Finance Brownbag, Economics Department Public/Labor Lunch Spring and Fall 2022, NBER Behavioral Public Economics Boot Camp  \\

\textbf{2021} Amir Sufi's Corporate Finance Reading Group, Chicago Booth Finance Brownbag, Economic Dynamics Working Group, G3 Finance Student Seminar, Student Applied Microeconomics Lunch \\

\textbf{2020} G3 Finance Student Seminar, Macrofinance Student Workshop

\section{Other\\ Employment\\ And Positions} \textbf{Princeton University} \\
Julis-Rabinowitz Center for Public Policy and Finance

Research Assistant to Atif Mian  \hfill 2017-2018 \\


\textbf{Federal Reserve Bank of St. Louis} \\
Research Division

Research Assistant to David Andolfatto and Guillaume Vandenbroucke  \hfill 2014-2016 \\

\section{Additional Information} \textbf{Citizenship:} USA \\
\textbf{Languages:} English, Spanish \\
\textbf{Computer Skills:} Stata, Matlab, R, Python


\vfill \hfill August 1 \\




\end{document}

%%%%%%%%%%%%%%%%%%%%%%%%%% End CV Document %%%%%%%%%%%%%%%%%%%%%%%%%%%%%
